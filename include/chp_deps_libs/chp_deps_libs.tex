\chapter{Caffe的依赖库}
Caffe代码的实现依赖了大量第三方功能库(特别是使用了多个Google公司的开源库),在“站在巨人的肩膀上”的同时,也为Caffe代码的学习带来一定困难。本章将对这些依赖库的功能和使用做简要介绍。
% Section X.1
\section{准标准库Boost}
Boost库定义在namespace boost中。
% Section X.1.1
\subsection{智能指针}\label{deps/boost/ptr}
\subsubsection{scoped\_ptr<T>}
scoped\_ptr<T>是Boost中最简单的智能指针,当指针离开其作用域时候,释放相关资源。需要特别注意的一点是scoped\_ptr<T>不能共享指针的所有权也不能转移所有权,因此该指针不能作为函数返回值使用。
\begin{minted}{c++}
class test { 
 public: 
  void print() { 
    cout << "test print now" <<endl; 
  } 
};

int _tmain(int argc, _TCHAR* argv[]) { 
  boost::scoped_ptr<test> x(new test); 
  x->print(); 
  return 0; 
}
\end{minted}
\subsubsection{shared\_ptr<T>}
shared\_ptr<T>是Boost智能指针中最常使用的指针,该指针中包含一个引用计数器,当类型为T的对象的引用数量为0时,该对象所在的堆内存会被自动释放。这里要注意的是该指针不能指向一块动态增长的内存(可以使用share\_array<T>代替)。
\begin{minted}{c++}
class test { 
 public: 
  void print() { 
    cout << "test print now" <<endl; 
  } 
};

int _tmain(int argc, _TCHAR* argv[]) { 
  boost::shared_ptr<test> ptr_1(new test); 
  ptr_1->print();//引用计数为1
  boost::shared_ptr<test> ptr_2 = ptr_1; 
  ptr_2->print();//引用计数为2
  ptr_1->print();//引用计数还是为2
  return 0; 
}
\end{minted}
\subsubsection{weak\_ptr<T>}
weak\_ptr是一个弱指针,它可以共享一个share\_ptr的内存,但是无论是构造还是析构一个weak\_ptr 都不会影响引用计数器。
\subsubsection{intrusive\_ptr<T>}
intrusive\_ptr和share\_ptr功能一样,但比share\_ptr效率高,不过intrusive\_ptr要求类型T本身维护一个引用计数器。
\subsubsection{scoped\_array<T> 和 shared\_array<T>}
shared\_ptr和scoped\_ptr不能用于数组的内存,所以shared\_array和scoped\_array可以作为他们的代替品。但是,这里还是建议使用std::vector<shared\_ptr>。
\begin{minted}{c++}
int _tmain(int argc, _TCHAR* argv[]) { 
  const int size = 10; 
  boost::shared_array<test> a(new test[]);
  for (int i = 0; i < size; ++i) 
    a[i].print();
  return 0; 
}
\end{minted}

% Section X.2
\section{日志功能glog库}\label{deps/glog}
% Section X.3
\section{程序参数处理gflags库}\label{deps/gflags}
% Section X.4
\section{单元测试gtest库}
% Section X.5
\section{结构化protobuffer库}\label{deps/protobuf}
全称为Protocol Buffers,是一种结构化数据存储格式,可以对存储数据进行序列化和反序列化,可用于网络中的结构化数据传输。在Caffe中,protobuffer主要用来存储和管理支持网络运行的各种相关参数,它可以将参数以对象的形式序列化后存储于二进制文件中,然后通过反序列化对其进行读取。
% Section X.5.1
\subsection{使用介绍}
% Section X.5.2
\subsection{Caffe中Protobuffer的结构}

% Section X.6
\section{图像处理opencv库}
类型:
cv::Size
cv::Scalar
方法:
cv::merge()
cv::mean()
cv::cvtColor()
cv::resize()
cv::Mat的convertTo()
cv::subtract()
cv::splite()
cv::imread()
cv::Mat(height, width, type, data)
cv::Mat(cv::Size, type, cv::Scalar)
% Section X.7
\section{代数运算BLAS}
BLAS全称Basic Linear Algebra Subprograms,是线性代数运算中所需算法的集合,其中包括诸如向量加法、向量乘法、矩阵乘法等。BLAS最早起源于Fortran,其中定义的函数接口后来被其他代数运算库做为标准接口使用,这些运算库包括:ATLAS、Intel Math Kernel Library (MKL)和OpenBLAS等。在NVIDIA公司的CUDA SDK中也包括了一个叫做cuBLAS的BLAS实现,主要应用于该公司的GPU产品。
% Section X.7.1
\subsection{CPU端的BLAS}\label{deps/blas/cpu}
\subsubsection{cblas\_*axpy()}
\begin{cnfrmfunc}
   \item{\kai{原型}}\\ 
     void cblas\_saxpy(const int n, const float a, const float* x, const int incx float* y, int incy) \\
     void cblas\_daxpy(const int n, const double a, const double* x, const int incx double* y, int incy)
   \item{\kai{参数}}\\
     n - x和y中元素个数\\
     a - 尺度系数\\
     x - 数组x首地址\\
     incx - x的计算间隔\\
     y - 数组y首地址\\
     incy- y的计算间隔
   \item{\kai{返回值}}\\
     float或double
   \item{\kai{描述}}\\
     $y\coloneqq a*x+y$
\end{cnfrmfunc}
\subsubsection{cblas\_*asum()}
\begin{cnfrmfunc}
   \item{\kai{原型}}\\ 
     float cblas\_sasum (const int n, const float *x, const int incx)\\
     double cblas\_dasum (const int n, const double *x, const int incx)
   \item{\kai{参数}}\\
     n - x和y中元素个数\\
     x - 数组x首地址\\
     incx - x的计算间隔
   \item{\kai{返回值}}\\
     float或者double
   \item{\kai{描述}}\\
     $res\coloneqq \sum\limits_{i}^{n}x[i]$
\end{cnfrmfunc}
\subsubsection{cblas\_*dot()}
\begin{cnfrmfunc}
   \item{\kai{原型}}\\ 
     float cblas\_sdot (const int n, const float *x, const int incx, const float *y, const int incy)\\
     double cblas\_ddot (const int n, const float *x, const int incx, const float *y, const int incy)
   \item{\kai{参数}}\\
     n - x和y中元素个数\\
     x - 数组x首地址\\
     incx - x的计算间隔\\
     y - 数组y的首地址\\
     incy - y的计算间隔
   \item{\kai{返回值}}\\
     float或者double
   \item{\kai{描述}}\\
     $res\coloneqq \sum\limits_{i}^{n}x[i]*y[i]$
\end{cnfrmfunc}
\subsubsection{cblas\_*scal()}
\begin{cnfrmfunc}
  \item{\kai{原型}}\\
    void cblas\_sscal (const int n, const float a, float *x, const int incx)\\
    void cblas\_dscal (const int n, const double a, double *x, const int incx)
   \item{\kai{参数}}\\
     n - x和y中元素个数\\
     a - 系数\\
     x - 数组x首地址\\
     incx - x的计算间隔\\
   \item{\kai{返回值}}\\
     空值
   \item{\kai{描述}}\\
     $res\coloneqq a * x[i]$
\end{cnfrmfunc}   
   
 % Section X.7.2
\subsection{GPU端的BLAS}\label{deps/blas/gpu}
\subsubsection{cublas*axpy()}
\begin{cnfrmfunc}
   \item{\kai{原型}}\\ 
     cublasStatus\_t cublasSaxpy(cublasHandle\_t handle, int n,
                           const float           *alpha,
                           const float           *x, int incx,
                           float                 *y, int incy)\\
     cublasStatus\_t cublasDaxpy(cublasHandle\_t handle, int n,
                           const double          *alpha,
                           const double          *x, int incx,
                           double                *y, int incy)
   \item{\kai{参数}}\\
     handle - cuBLAS库的句柄\\                      
     n - x和y中元素个数\\
     alpha - 尺度系数\\
     x - 数组x首地址\\
     incx - x的计算间隔\\
     y - 数组y首地址\\
     incy- y的计算间隔
   \item{\kai{返回值}}\\
     cublasStatus\_t
   \item{\kai{描述}}\\
     $y\coloneqq a*x+y$
\end{cnfrmfunc}
\subsubsection{cublas*asum()}
\begin{cnfrmfunc}
   \item{\kai{原型}}\\ 
     cublasStatus\_t  cublasSasum(cublasHandle\_t handle, int n,
                                 const float *x, int incx, float  *result)\\
     cublasStatus\_t  cublasDasum(cublasHandle\_t handle, int n,
                                 const double *x, int incx, double *result)
   \item{\kai{参数}}\\
     handle - cuBLAS库的句柄\\                      
     n - x中元素个数\\
     x - 数组x首地址\\
     incx - x的计算间隔\\
     result - 保存计算的结果
   \item{\kai{返回值}}\\
     cublasStatus\_t
   \item{\kai{描述}}\\
     $result\coloneqq \sum\limits_{i}^{n}x[i]$
\end{cnfrmfunc}
\subsubsection{cublas*dot()}
\begin{cnfrmfunc}
   \item{\kai{原型}}\\ 
     cublasStatus\_t cublasSdot (cublasHandle\_t handle, int n,
                                const float *x, int incx,
                                const float *y, int incy,
                                float *result)\\
     cublasStatus\_t cublasDdot (cublasHandle\_t handle, int n,
                                const double *x, int incx,
                                const double *y, int incy,
                                double *result)
   \item{\kai{参数}}\\
     handle - cuBLAS库的句柄\\                      
     n - x中元素个数\\
     x - 数组x首地址\\
     incx - x的计算间隔\\
     y - 数组y首地址\\
     incy - y的计算间隔\\
     result - 保存计算的结果
   \item{\kai{返回值}}\\
     cublasStatus\_t
   \item{\kai{描述}}\\
     $result\coloneqq \sum\limits_{i}^{n}x[i]*y[i]$
\end{cnfrmfunc}
\subsubsection{cublas*scal()}
\begin{cnfrmfunc}
   \item{\kai{原型}}\\ 
     cublasStatus\_t  cublasSscal(cublasHandle\_t handle, int n,
                                 const float *alpha,
                                 float *x, int incx)\\
     cublasStatus\_t  cublasDscal(cublasHandle\_t handle, int n,
                                 const double *alpha,
                                 double *x, int incx)
   \item{\kai{参数}}\\
     handle - cuBLAS库的句柄\\                      
     n - x中元素个数\\
     alpha - 系数\\
     x - 数组x首地址\\
     incx - x的计算间隔\\
   \item{\kai{返回值}}\\
     cublasStatus\_t
   \item{\kai{描述}}\\
     $x[i]\coloneqq alpha * x[i]$
\end{cnfrmfunc}
   

