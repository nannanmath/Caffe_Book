\chapter{CUDA辅助宏}
Caffe中定义了一些列的宏和函数对CUDA相关操作做了一定的处理,这些宏和函数一头文件的形式保存在device\_alternate.hpp中,下边将对其做简要介绍。
% Section X.1
\section{GPU模式相关}\label{cudamacro/gpumode}
\subsubsection{NO\_GPU宏}
该宏一般在检测到程序没有启用GPU模式时被调用,宏内部使用了glog库,显示一段字符串标明没有使用GPU模式。
\begin{minted}{c++}
#define NO_GPU LOG(FATAL) << "Cannot use GPU in CPU-only Caffe: check mode."
\end{minted}
% Section X.2
\section{CUDA返回错误处理}\label{cudamacro/err}
\subsubsection{CUDA\_CHECK()宏}
检测CUDA函数的返回值是否出错,通过glog库的CHECK宏判断返回值是否为error,如果为error则调用cudaGetErrorString()(参考\ref{cuda/err})函数输出错误信息。
\begin{minted}{c++}
#define CUDA_CHECK(condition) \
  /* Code block avoids redefinition of cudaError_t error */ \
  do { \
    cudaError_t error = condition; \
    CHECK_EQ(error, cudaSuccess) << " " << cudaGetErrorString(error); \
  } while (0)
\end{minted}
\subsubsection{CUBLAS\_CHECK()宏}
检测CUDA函数的返回值是否出错,通过glog库的CHECK宏判断返回值是否为error,如果为error则调用cublasGetErrorString()(参考\ref{common/err/cublas})函数输出错误信息。
\begin{minted}{c++}
#define CUBLAS_CHECK(condition) \
  do { \
    cublasStatus_t status = condition; \
    CHECK_EQ(status, CUBLAS_STATUS_SUCCESS) << " " \
      << caffe::cublasGetErrorString(status); \
  } while (0)
\end{minted}
\subsubsection{CURAND\_CHECK()宏}
检测CUDA函数的返回值是否出错,通过glog库的CHECK宏判断返回值是否为error,如果为error则调用curandGetErrorString()(参考\ref{common/err/curand})函数输出错误信息。
\begin{minted}{c++}
#define CURAND_CHECK(condition) \
  do { \
    curandStatus_t status = condition; \
    CHECK_EQ(status, CURAND_STATUS_SUCCESS) << " " \
      << caffe::curandGetErrorString(status); \
  } while (0)
\end{minted}

% Section X.3
\section{CUDA线程相关}
