\section{训练与测试caffe.cpp}
caffe.cpp实现了一个网络的训练与测试功能,是Caffe中最常用的程序。该程序中包含很多技巧,其中一些很值得在软件开发中借鉴使用。


首先判断是否启用python\_layer:
\begin{minted}{c++}
#ifdef WITH_PYTHON_LAYER
#include "boost/python.hpp"
namespace bp = boost::python;
#endif
\end{minted}
通过Makefile判断是否启用python\_layer,如果启用则包含Boost库的boost::python库,实现python调用C++功能。


使用gflags进行参数设置:
\begin{minted}{c++}
DEFINE_string(gpu, "",
    "Optional; run in GPU mode on given device IDs separated by ','."
    "Use '-gpu all' to run on all available GPUs. The effective training "
    "batch size is multiplied by the number of devices.");
DEFINE_string(solver, "",
    "The solver definition protocol buffer text file.");
DEFINE_string(model, "",
    "The model definition protocol buffer text file.");
DEFINE_string(snapshot, "",
    "Optional; the snapshot solver state to resume training.");
DEFINE_string(weights, "",
    "Optional; the pretrained weights to initialize finetuning, "
    "separated by ','. Cannot be set simultaneously with snapshot.");
DEFINE_int32(iterations, 50,
    "The number of iterations to run.");
DEFINE_string(sigint_effect, "stop",
             "Optional; action to take when a SIGINT signal is received: "
              "snapshot, stop or none.");
DEFINE_string(sighup_effect, "snapshot",
             "Optional; action to take when a SIGHUP signal is received: "
             "snapshot, stop or none.");
\end{minted}
其中包含的参数有:gpu,solver,model,snapshot,weights,iteration,sigint\_effect和sighup\_effect。主要功能见表\ref{chp-from-tools-caffe-cpp1}:
\begin{cntable}{参数说明}{chp-from-tools-caffe-cpp1}
  \begin{tabular}{|c|l|}
    \hline
    gpu & 是否使用GPU运行 \\ \hline
    solver & solver文件路径 \\ \hline
    model & 模型路径 \\ \hline
    snapshot & 参数文件保存路径 \\ \hline
    weights & 参数文件路径 \\ \hline
    iterations & 最大迭代次数 \\ \hline
    sigint\_effect & 不知道 \\ \hline
    sighup\_effect & 不知道 \\
    \hline
  \end{tabular}
\end{cntable}