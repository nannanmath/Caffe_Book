\chapter{数学运算及辅助函数}
Caffe将底层的数学运算和一些辅助函数专门进行了分离,统一保存在名为math\_functions.hpp、math\_functions.cpp和math\_functions.cu的文件中,其中math\_functions.cu是GPU上运行的函数实现。
% Section X.1
\section{数学运算函数}
% Section X.1.1
\subsection{CPU函数}
% Section X.1.2
\subsection{GPU函数}
% Section X.2
\section{存储操作}
% Section X.2.1
\subsection{内存初始化}\label{math/mem/init}
\subsubsection{caffe\_memset()}
\begin{cnfrmfunc}
   \item{\kai{原型}}\\
     inline void caffe\_memset(const size\_t N, const int alpha, void* X)
   \item{\kai{参数}}\\
     N - 字节数\\
     alpha - 初始值\\
     X - 内存首地址
   \item{\kai{返回值}}\\
     空值
\end{cnfrmfunc}
该函数内部将调用memset()函数进行内存赋值。
\subsubsection{caffe\_gpu\_memset()}
\begin{cnfrmfunc}
   \item{\kai{原型}}\\
     inline void caffe\_gpu\_memset(const size\_t N, const int alpha, void* X)
   \item{\kai{参数}}\\
     N - 字节数\\
     alpha - 初始值\\
     X - 内存首地址
   \item{\kai{返回值}}\\
     空值
\end{cnfrmfunc}
该函数内部将调用cudaMemset()函数进行内存赋值(参考~\ref{cuda/mem/init})。
% Section X.2.2
\subsection{内存拷贝}\label{math/mem/cpy}
\subsubsection{caffe\_gpu\_memcpy()}
\begin{cnfrmfunc}
   \item{\kai{原型}}\\
     void caffe\_gpu\_memcpy(const size\_t N, const void* X, void* Y)
   \item{\kai{参数}}\\
     N - 拷贝字节数\\
     X - 待拷贝内存首地址(设备端)\\
     Y - 拷贝目的地址(主机端)
   \item{\kai{返回值}}\\
     空值
\end{cnfrmfunc}
该函数将数据从设备端拷贝至主机端。函数具体实现如下:
\begin{minted}{c++}
void caffe_gpu_memcpy(const size_t N, const void* X, void* Y) {
  if (X != Y) {
    CUDA_CHECK(cudaMemcpy(Y, X, N, cudaMemcpyDefault));  // NOLINT(caffe/alt_fn)
  }
}
\end{minted}
通过调用cudaMemcpy()函数实现(参考~\ref{cuda/mem/init}),注意这里的拷贝目的地址Y与使用统一地址空间的设备之间可以分享,函数最后一个参数设置为cudaMemcpyDefault。