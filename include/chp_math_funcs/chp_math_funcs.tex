\chapter{数学运算及辅助函数}
Caffe将底层的数学运算和一些辅助函数专门进行了分离,统一保存在名为math\_functions.hpp、math\_functions.cpp和math\_functions.cu的文件中,其中math\_functions.cu是GPU上运行的函数实现。
% Section X.1
\section{数学运算函数}
% Section X.1.1
\subsection{CPU函数}\label{math/cpu/alg}
\subsubsection{caffe\_axpy()}
\begin{cnfrmfunc}
   \item{\kai{原型}}\\
     template <>\\
     void caffe\_axpy<float>(const int N, const float alpha, const float* X, float* Y)\\
     
     template <>\\
     void caffe\_axpy<double>(const int N, const float alpha, const float* X, float* Y)
   \item{\kai{参数}}\\
     N - 元素个数\\
     alpha - 系数\\
     X - 数组X首地址\\
     Y - 数组Y首地址
   \item{\kai{返回值}}\\
     空值
\end{cnfrmfunc}
该函数通过调用cblas\_saxpy()和cblas\_daxpy()函数(参考\ref{deps/blas/cpu})在主机端完成如下所示运算:
$$
Y[i] = \alpha * X[i] + Y[i]
$$
\subsubsection{caffe\_cpu\_asum()}
\begin{cnfrmfunc}
   \item{\kai{原型}}\\
     template <>\\
     float caffe\_cpu\_asum<float>(const int n, const float* x)\\

     template <>\\
     double caffe\_cpu\_asum<double>(const int n, const double* x)
   \item{\kai{参数}}\\
     n - 元素个数\\
     x - 数组x首地址
   \item{\kai{返回值}}\\
     float或double
\end{cnfrmfunc}
该函数通过调用cblas\_sasum()和cblas\_dasum()函数(参考\ref{deps/blas/cpu})在主机端完成如下所示运算:
$$
return = \sum\limits_{i}^{n} x[i]
$$
\subsubsection{caffe\_cpu\_strided\_dot()}
\begin{cnfrmfunc}
   \item{\kai{原型}}\\
     template <>\\
     float caffe\_cpu\_strided\_dot<float>(const int n, const float* x, const int incx, const float* y, const int incy)\\

     template <>\\
     double caffe\_cpu\_strided\_dot<double>(const int n, const double* x, const int incx, const double* y, const int incy)
   \item{\kai{参数}}\\
     n - 元素个数\\
     x - 数组x首地址\\
     incx - 数组x的计算间隔\\
     y - 数组y首地址\\
     incy - 数组y的计算间隔
   \item{\kai{返回值}}\\
     float或double
\end{cnfrmfunc}
该函数通过调用此blas\_sdot()和cblas\_ddot()函数(参考\ref{deps/blas/cpu})在主机端完成如下所示运算:
$$
return = \sum\limits_{i}^{n} x[i]*y[i]
$$
\subsubsection{caffe\_cpu\_dot()}
\begin{cnfrmfunc}
   \item{\kai{原型}}\\
     template <typename Dtype>\\
     Dtype caffe\_cpu\_dot(const int n, const Dtype* x, const Dtype* y)\\

     template\\
     float caffe\_cpu\_dot<float>(const int n, const float* x, const float* y)\\

     template\\
     double caffe\_cpu\_dot<double>(const int n, const double* x, const double* y)
   \item{\kai{参数}}\\
     n - 元素个数\\
     x - 数组x首地址\\
     y - 数组y首地址
   \item{\kai{返回值}}\\
     float或double
\end{cnfrmfunc}
该函数通过调用caffe\_cpu\_strided\_dot()函数(该函数依然为Caffe自定义的函数)在主机端完成如下所示运算:
$$
return = \sum\limits_{i}^{n} x[i]*y[i]
$$
\subsubsection{caffe\_scal()}
\begin{cnfrmfunc}
   \item{\kai{原型}}\\
     template <>\\
     void caffe\_scal<float>(const int N, const float alpha, float *X)\\

     template<>\\
     void caffe\_scal<double>(const int N, const double alpha, double *X)
   \item{\kai{参数}}\\
     N - 元素个数\\
     alpha - 数组x首地址\\
     X - 数组X首地址
   \item{\kai{返回值}}\\
     空值
\end{cnfrmfunc}
该函数通过调用cblas\_sscal()和cblas\_dscal()函数(参考\ref{deps/blas/cpu})在主机端完成如下所示运算:
$$
X[i] = alpha * X[i]
$$
\subsubsection{caffe\_copy()}
\begin{cnfrmfunc}
   \item{\kai{原型}}\\
     template <typename Dtype>\\
     void caffe\_copy(const int N, const Dtype* X, Dtype* Y)\\
  
     template void caffe\_copy<int>(const int N, const int* X, int* Y)\\
     template void caffe\_copy<unsigned int>(const int N, const unsigned int* X, unsigned int* Y)\\
     template void caffe\_copy<float>(const int N, const float* X, float* Y)\\
     template void caffe\_copy<double>(const int N, const double* X, double* Y)
   \item{\kai{参数}}\\
     N - 元素个数\\
     X - 数组X首地址\\
     Y - 数组Y首地址
   \item{\kai{返回值}}\\
     空值
\end{cnfrmfunc}
该函数内部通过调用memcpy()和cudaMemcpy()函数在主机端和设备端完成拷贝操作。

% Section X.1.2
\subsection{GPU函数}\label{math/gpu/alg}
\subsubsection{caffe\_gpu\_axpy()}
\begin{cnfrmfunc}
   \item{\kai{原型}}\\
     template <>\\
     void caffe\_gpu\_axpy<float>(const int N, const float alpha, const float* X, float* Y) \\

     template <>\\
     void caffe\_gpu\_axpy<double>(const int N, const float alpha, const float* X, float* Y)
   \item{\kai{参数}}\\
     N - 元素个数\\
     alpha - 系数\\
     X - 数组X首地址\\
     Y - 数组Y首地址
   \item{\kai{返回值}}\\
     空值
\end{cnfrmfunc}
该函数通过调用cublasSaxpy()和cublasDaxpy()函数(参考\ref{deps/blas/gpu})在设备端完成如下所示运算:
$$
Y[i] = \alpha * X[i] + Y[i]
$$
\subsubsection{caffe\_gpu\_asum()}
\begin{cnfrmfunc}
   \item{\kai{原型}}\\
     template <>\\
     void caffe\_gpu\_asum<float>(const int n, const float* x, float* y)\\

     template <>\\
     void caffe\_gpu\_asum<double>(const int n, const double* x, double* y)
   \item{\kai{参数}}\\
     n - 元素个数\\
     x - 数组x首地址\\
     y - 存储计算结果
   \item{\kai{返回值}}\\
     空值
\end{cnfrmfunc}
该函数通过调用cublasSasum()和cublasDasum()函数(参考\ref{deps/blas/gpu})在设备端完成如下所示运算:
$$
y = \sum\limits_{i}^{n}x[i]
$$
\subsubsection{caffe\_gpu\_dot()}
\begin{cnfrmfunc}
   \item{\kai{原型}}\\
     template <>\\
     void caffe\_gpu\_dot<float>(const int n, const float* x, const float* y, float* out)\\

     template <>\\
     void caffe\_gpu\_dot<double>(const int n, const double* x, const double* y, double* out) 
   \item{\kai{参数}}\\
     n - 元素个数\\
     x - 数组x首地址\\
     y - 数组y首地址\\
     out - 存储计算结果
   \item{\kai{返回值}}\\
     空值
\end{cnfrmfunc}
该函数通过调用cublasSdot()和cublasDdot()函数(参考\ref{deps/blas/gpu})在设备端完成如下所示运算:
$$
out = \sum\limits_{i}^{n}x[i]*y[i]
$$
\subsubsection{caffe\_gpu\_scal()}
\begin{cnfrmfunc}
   \item{\kai{原型}}\\
     template <>\\
     void caffe\_gpu\_scal<float>(const int N, const float alpha, float *X)\\

     template <>\\
     void caffe\_gpu\_scal<double>(const int N, const double alpha, double *X)
   \item{\kai{参数}}\\
     N - 元素个数\\
     alpha - 系数\\
     X - 数组X首地址\\
   \item{\kai{返回值}}\\
     空值
\end{cnfrmfunc}
该函数通过调用cublasSscal()和cublasDscal()函数(参考\ref{deps/blas/gpu})在设备端完成如下所示运算:
$$
X[i] = alpha * X[i]
$$

% Section X.2
\section{存储操作}
% Section X.2.1
\subsection{内存初始化}\label{math/mem/init}
\subsubsection{caffe\_memset()}
\begin{cnfrmfunc}
   \item{\kai{原型}}\\
     inline void caffe\_memset(const size\_t N, const int alpha, void* X)
   \item{\kai{参数}}\\
     N - 字节数\\
     alpha - 初始值\\
     X - 内存首地址
   \item{\kai{返回值}}\\
     空值
\end{cnfrmfunc}
该函数内部将调用memset()函数进行内存赋值。
\subsubsection{caffe\_gpu\_memset()}
\begin{cnfrmfunc}
   \item{\kai{原型}}\\
     inline void caffe\_gpu\_memset(const size\_t N, const int alpha, void* X)
   \item{\kai{参数}}\\
     N - 字节数\\
     alpha - 初始值\\
     X - 内存首地址
   \item{\kai{返回值}}\\
     空值
\end{cnfrmfunc}
该函数内部将调用cudaMemset()函数进行内存赋值(参考~\ref{cuda/mem/init})。
% Section X.2.2
\subsection{内存拷贝}\label{math/mem/cpy}
\subsubsection{caffe\_gpu\_memcpy()}
\begin{cnfrmfunc}
   \item{\kai{原型}}\\
     void caffe\_gpu\_memcpy(const size\_t N, const void* X, void* Y)
   \item{\kai{参数}}\\
     N - 拷贝字节数\\
     X - 待拷贝内存首地址(设备端)\\
     Y - 拷贝目的地址(主机端)
   \item{\kai{返回值}}\\
     空值
\end{cnfrmfunc}
该函数将数据从设备端拷贝至主机端。函数具体实现如下:
\begin{minted}{c++}
void caffe_gpu_memcpy(const size_t N, const void* X, void* Y) {
  if (X != Y) {
    CUDA_CHECK(cudaMemcpy(Y, X, N, cudaMemcpyDefault));  // NOLINT(caffe/alt_fn)
  }
}
\end{minted}
通过调用cudaMemcpy()函数实现(参考~\ref{cuda/mem/init}),注意这里的拷贝目的地址Y与使用统一地址空间的设备之间可以分享,函数最后一个参数设置为cudaMemcpyDefault。