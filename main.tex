\documentclass{NanCNBook}

\title{\hei\chuhao{Caffe\cite{jia2014caffe}解析}}
\author{nannanmath}

\begin{document}

\maketitle{}
\thispagestyle{empty}

\frontmatter
\pagestyle{plain}
\chapter*{前{\quad}言}
学习Caffe\footnote{http://caffe.berkeleyvision.org/}的设计与实现的细节,从而更好的实现机器学习相关的研究。



%\pagenumbering{roman}
\tableofcontents
\listoffigures
\listoftables


%\pagenumbering{arabic}
\mainmatter
\pagestyle{fancy}

\part{Caffe的依赖与结构}

\chapter{Caffe的依赖库}
Caffe代码的实现依赖了大量第三方功能库(特别是使用了多个Google公司的开源库),在“站在巨人的肩膀上”的同时,也为Caffe代码的学习带来一定困难。本章将对这些依赖库的功能和使用做简要介绍。
\section{准标准库Boost}

\section{日志功能glog库}

\section{程序参数处理gflags库}

\section{单元测试gtest库}

\section{结构化protobuf库}

\section{图像处理opencv库}

\chapter{Caffe的层次}

\part{Caffe的外部调用}

\chapter{从tools开始}
在Caffe代码的tools目录下包含了诸多作为tools的工具软件代码,他们通过对Caffe核心功能的调用实现各种外层的逻辑应用。

\section{计算图像均值compute\_image\_mean.cpp}

\section{图像写入数据库convert\_query.cpp}

\section{训练与测试caffec.cpp}

\section{特征提取extract\_feature.cpp}

\section{网络微调finetune\_net.cpp}

\part{Caffe的内部实现}
\chapter{底层数据结构}

\chapter{网络中的数据Blob类}

\chapter{网络定义Layer类}

\chapter{网络管理Net类}

\chapter{网络的运行Slover类}

%%%%%%%%%%%%%%%%%%%%%%%%%%
%%%%%%%%%%%%%%%%%%%%%%%%%
\cnbibliography{bibtex.bib}


% \appendix
\begin{cnappendix}
\chapter{编译指挥官Makefile}

\end{cnappendix}

\end{document}


%%% Local Variables:
%%% mode: latex
%%% TeX-master: t
%%% End:
