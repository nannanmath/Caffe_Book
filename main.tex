\documentclass{NanCNBook}

%\includeonly{}


\title{\hei\chuhao{Caffe\cite{jia2014caffe}解析}}
\author{nannanmath}

\begin{document}

\maketitle{}
\thispagestyle{empty}

\frontmatter
\pagestyle{plain}
\chapter*{前{\quad}言}
学习Caffe\footnote{http://caffe.berkeleyvision.org/}的设计与实现的细节,从而更好的实现机器学习相关的研究。



%\pagenumbering{roman}
\tableofcontents
\listoffigures
\listoftables


%\pagenumbering{arabic}
\mainmatter
\pagestyle{fancy}

\part{Caffe的依赖与结构}

\chapter{Caffe的依赖库}
Caffe代码的实现依赖了大量第三方功能库(特别是使用了多个Google公司的开源库),在“站在巨人的肩膀上”的同时,也为Caffe代码的学习带来一定困难。本章将对这些依赖库的功能和使用做简要介绍。
\section{准标准库Boost}

\section{日志功能glog库}

\section{程序参数处理gflags库}

\section{单元测试gtest库}

\section{结构化protobuf库}

\section{图像处理opencv库}

\chapter{Caffe的层次}

\part{Caffe的外部调用}

\chapter{从tools开始}
在Caffe代码的tools目录下包含了诸多作为tools的工具软件代码,他们通过对Caffe核心功能的调用实现各种外层的逻辑应用。

\section{计算图像均值compute\_image\_mean.cpp}

\section{图像写入数据库convert\_query.cpp}

\section{训练与测试caffe.cpp}
caffe.cpp实现了一个网络的训练与测试功能,是Caffe中最常用的程序。该程序中包含很多技巧,其中一些很值得在软件开发中借鉴使用。


首先判断是否启用python\_layer:
\begin{minted}{c++}
#ifdef WITH_PYTHON_LAYER
#include "boost/python.hpp"
namespace bp = boost::python;
#endif
\end{minted}
通过Makefile判断是否启用python\_layer,如果启用则包含Boost库的boost::python库,实现python调用C++功能。


使用gflags进行参数设置:
\begin{minted}{c++}
DEFINE_string(gpu, "",
    "Optional; run in GPU mode on given device IDs separated by ','."
    "Use '-gpu all' to run on all available GPUs. The effective training "
    "batch size is multiplied by the number of devices.");
DEFINE_string(solver, "",
    "The solver definition protocol buffer text file.");
DEFINE_string(model, "",
    "The model definition protocol buffer text file.");
DEFINE_string(snapshot, "",
    "Optional; the snapshot solver state to resume training.");
DEFINE_string(weights, "",
    "Optional; the pretrained weights to initialize finetuning, "
    "separated by ','. Cannot be set simultaneously with snapshot.");
DEFINE_int32(iterations, 50,
    "The number of iterations to run.");
DEFINE_string(sigint_effect, "stop",
             "Optional; action to take when a SIGINT signal is received: "
              "snapshot, stop or none.");
DEFINE_string(sighup_effect, "snapshot",
             "Optional; action to take when a SIGHUP signal is received: "
             "snapshot, stop or none.");
\end{minted}
其中包含的参数有:gpu,solver,model,snapshot,weights,iteration,sigint\_effect和sighup\_effect。主要功能见表\ref{tab:参数说明}:
\begin{table}[h]
  \setlength{\abovecaptionskip}{-5pt}
  \caption{参数说明}
  \centering
  \begin{tabular}{|c|l|}
    \hline
    gpu & 是否使用GPU运行 \\ \hline
    solver & solver文件路径 \\ \hline
    model & 模型路径 \\ \hline
    snapshot & 参数文件保存路径 \\ \hline
    weights & 参数文件路径 \\ \hline
    iterations & 最大迭代次数 \\ \hline
    sigint\_effect & 不知道 \\ \hline
    sighup\_effect & 不知道 \\
    \hline
  \end{tabular}
  \label{tab:参数说明}
\end{table}
                    



\section{特征提取extract\_feature.cpp}

\section{网络微调finetune\_net.cpp}

\part{Caffe的内部实现}
\chapter{底层数据结构}

\chapter{网络中的数据Blob类}

\chapter{网络定义Layer类}

\chapter{网络管理Net类}

\chapter{网络的运行Slover类}

%%%%%%%%%%%%%%%%%%%%%%%%%%
%%%%%%%%%%%%%%%%%%%%%%%%%
\cnbibliography{bibtex.bib}


% \appendix
\begin{cnappendix}
\chapter{编译指挥官Makefile}

\end{cnappendix}

\end{document}


%%% Local Variables:
%%% mode: latex
%%% TeX-master: t
%%% End:
